\documentclass[11pt,letterpaper]{article}

\usepackage{fancyhdr}
\usepackage[latin1]{inputenc}
\usepackage{amsmath}
\usepackage{amsfonts}
\usepackage{amssymb}
\usepackage{graphicx}
\usepackage[hmargin=2cm,vmargin=2.5cm]{geometry}
\usepackage[normalem]{ulem}
\usepackage{enumerate}
\usepackage{hyperref}



\pagestyle{fancy}
\setlength\parindent{0in}
\setlength\parskip{0.1in}
\setlength\headheight{15pt}


\begin{document}

\begin{flushright}
	\begin{Large}
		Miniproject 6: Applied Optimization
	\end{Large}
\end{flushright}

\noindent
\textbf{Overview:} This miniproject focuses on a central application of calculus, namely \emph{applied optimization}. These problems augment and extend the kinds of problems you have worked with in WeBWorK and class discussions. 

\medskip

\noindent
\textbf{Prerequisites:} Section 3.4 of \emph{Active Calculus.}

\medskip

\noindent
\textbf{This miniproject is one of the four designated CORE miniprojects.} Successful completion of this miniproject with a \textbf{Mastery} rating is required to earn a C or above in the course. 
	

\hrulefill

For this miniproject, select EXACTLY TWO of the following and give complete and correct solutions that abide by the specifications for student work. 

\begin{description}
	\item[Problem 1.] Two vertical towers of heights 60 ft and 80 ft stand on level ground, with their bases 100 ft apart. A cable that is stretched from the top of one pole to some point on the ground between the poles, and then to the top of the other pole. What is the minimum possible length of cable required? Justify your answer completely using calculus.
	\item[Problem 2.] Use calculus to find the point $(x,y)$ on the parabola traced out by $y = x^2$ that is closest to the point $(3,0)$. 
	\item[Problem 3.] A company is designing propane tanks that are cylindrical with hemispherical ends. Assume that the company wants tanks that will hold 1000 cubic feet of gas, and that the ends are more expensive to make, costing \$5 per square foot, while the cylindrical barrel between the ends costs \$2 per square foot. Use calculus to determine the minimum cost to construct such a tank.
\end{description}


\hrulefill

\noindent
\textbf{Submission instructions:} Please prepare a writeup that includes all your work for two of the above problems. Remember to begin each solution with a clear statement of the problem. 

The resulting writeup can be done in whatever fashion you wish but it must be saved as a PDF file and submitted using Blackboard. (You may use any program you want to write the writeup but the submission \emph{must} be a PDF, or your work will be marked at Novice level and returned without comment.) 

\noindent
\textbf{File name:} Please give your PDF file for this miniproject the title: 
\begin{verbatim}
	LastName Miniproject6.pdf
\end{verbatim}
where \texttt{LastName} is your last name. 

\end{document}