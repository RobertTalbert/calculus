\documentclass[11pt]{article}

\pagestyle{empty}                       %no page numbers
\thispagestyle{empty}                   %removes first page number
\setlength{\parindent}{0in}               %no paragraph indents

\usepackage{fullpage}
\usepackage[tmargin = 0.5in, bmargin = 1in, hmargin = 1in]{geometry}     %1-inch margins
\geometry{letterpaper}                  
\usepackage{graphicx}
\usepackage{amssymb}

% Default packages
\usepackage{latexsym}
\usepackage{amsfonts}
\usepackage{amsmath}
\usepackage{amsthm}
\usepackage{palatino}
\usepackage{hyperref}
\usepackage{multicol}
\usepackage{fancyhdr}
\usepackage{enumitem}
\usepackage{pifont}

\newcommand{\cuthere}{%
\noindent
\raisebox{-2.8pt}[0pt][0.75\baselineskip]{\small\ding{34}}
\unskip{\tiny\dotfill}
}

\def\ra{\rightarrow}

\def\pageturn{\vfill 
\begin{flushright}
	\begin{small}
		Continued $\ra$
	\end{small}
\end{flushright} \newpage}


\begin{document}
	
	\thispagestyle{empty}
	\renewcommand{\headrulewidth}{0.0pt}
	\thispagestyle{fancy}
	\lhead{Prof. Talbert}
	\chead{MTH 201: Calculus 1}
	\rhead{August 29/30, 2013}
	\lfoot{}
	\cfoot{}
	\rfoot{}	
	
	\vspace*{0in}

		\begin{center}
			\begin{large}
			\textbf{Class Activities: Velocity} \\
			\end{large}
		\end{center}
	
	

Today we'll start by working through parts of Activity 1.1 in Section 1.1, followed by a quick debrief. Then we'll do the following, which is a remixed version of Activity 1.3. 

\bigskip

Consider the function $s(t) = 64 - 16(t-1)^2$, which gives the position of an object at time $t$. 


\begin{enumerate}
			\item Calculate the value of $s(2)$. 
			
			\vspace{0.5in}
			
			\item Calculate the expression $s(2+h)$ and simplify completely. 
			
			\vspace{1.3in}
			
			
			\item Calculate the value of $s(2+h) - s(2)$ and simplify completely, including factoring out any common factors. 
			
			\vspace{1.3in}
			
			\item Set up and simplify the expression for the average velocity of an object with position $s(t)$ on the interval $[2, 2+h]$. The result should be an expression whose only variable is $h$, and there should be no fractions in the result. 
			
\vfill 
\cuthere
\vspace*{1in}
			
\newpage
			
			\item Use the result of (d) to find the average velocity of an object with position $s(t)$ on the interval $[2, 2.5]$ and then on the interval $[1.9, 2]$. (Hint: What is the value of $h$ each time?)
			
			\vspace{1.5in}
			
			\item In your own words, what is happening as $h$ approaches $0$? 
			 
			
			\vspace{1in}
			
			\item Use the result of (4) to find the \emph{instantaneous} velocity of an object with position $s(t)$ at $t = 2$. What does the sign (positive/negative) of the answer indicate? And what are the units of the answer? 
	
	
		\vspace{1.5in}
		
		\item How would you adjust the above processes if you were using the same position function for $s$ but wanted to find the instantaneous velocity at $t = 1$? And by just looking at the graph, what do you think is the value of the instantaneous velocity at $t=1$? 
	
	
	
\end{enumerate}

\vfill

\cuthere

\noindent
\textbf{What was the muddiest point from today's class?}

\vspace{1in}

\end{document}