\documentclass[11pt]{article}
\usepackage{amsmath}
\usepackage{amssymb}
\usepackage{enumitem}
\usepackage{fancyhdr}
\usepackage{palatino}

\pagestyle{fancy}


\lhead{MTH 201 (Calculus)}
\chead{Module 1a Post-Class}
\rhead{Talbert}

\begin{document}

\begin{flushleft}
        \Large{MTH 201: Module 1a Post-Class Activity}
\end{flushleft}

\begin{flushleft}
    \textbf{Due:} On Blackboard by 11:59pm Eastern, Friday September 4. \\
    \textbf{Time estimate:} 45-60 minutes. 
\end{flushleft}

\subsection*{Instructions}

Since our class is a \textbf{hybrid} course, we will not have a second class meeting over the material in Module 1a. Instead, you'll apply, analyze, and evaluate the material we learned in class by doing this \textbf{post-class} activity that combines work done independently with some online group work. This activity is the equivalent of what we would do in class if we had a second class meeting, and it should not take you much more than the same amount of time to complete. \textbf{If you find yourself spending more than an hour on this activity and feel you are going nowhere, STOP and ask a question.} 





\subsection*{Activities to complete}

\begin{enumerate}
    \item Read through the text just before Example 1.1.2 and then read through the solution of Example 1.1.2. Answer the following questions: 
        \begin{enumerate}
            \item The text before Example 1.1.2 talks about a variable called $h$. What does $h$ represent, in everyday terms --- and why should we investigate what happens as $h$ approaches zero? 
            \item In part (a) of the solution to Example 1.1.2, the following expression is given: 
            \begin{equation*}
AV_{[0.5, 0.5+h]} = \frac{s(0.5+h) - s(0.5)}{(0.5+h) - 0.5}\text{.}
\end{equation*}
            Explain in your own words how the author got this expression. 
            \item Also in the solution for part (a), there's this derivation: 
\begin{align*}
s(0.5+h) &=  16 - 16(0.5 + h)^2\\
&=  16 - 16(0.25 + h + h^2)\\
&=  16 - 4 - 16h - 16h^2\\
&=  12 - 16h - 16h^2\text{.}
\end{align*}
        There are four ``equals'' signs in this computation. Explain in your own words how each equality was obtained. That is, first explain why $s(0.5+h)$ equals  $16 - 16(0.5 + h)^2$; then explain why $16 - 16(0.5 + h)^2$ equals $16 - 16(0.25 + h + h^2)$; and so on. 
            \item Read through the rest of the derivations in the solution to part (a). Notice that the author claims that $\dfrac{-16h-16h^2}{h}$ is equal to $-16 - 16h$. How did the author get that result? Why isn't $\dfrac{-16h-16h^2}{h}$ instead equal to $-16-16h^2$? 
            \item The solution goes on to say that because the average velocity from $t=0.5$ to $t=0.5+h$ is $-16 - 16h$, then the instantaneous velocity of the ball at $t=0.5$ should be $-16$ feet per second. Explain in your own words how the author got that result. 
        \end{enumerate}
        
    \item Now work through Activity 1.1.4, which asks you essentially to replicate the solution for Example 1.1.2 but using a different function and a different point in time. Do this in two stages in order to really master the concepts: First, work out as much of the solution as possible by yourself using Example 1.1.2 as a guide. Then, once you have worked the solution out by yourself, you will be placed into a group of students from your ``team'' in the course and given a private chat area on CampusWire. Share your work with your group-mates in the private chat and discuss your work, correct errors, and ask questions. \textbf{The answers to this Activity are as follows. You and your group supply the work and explanations.}
    \begin{itemize}
        \item Most simplified expression for average velocity on $[2, 2+h]$: $-16h -32$
        \item Average velocity on $[1.5, 2]$: $-24$ feet per second
        \item Instantaneous velocity at $t=2$: $-32$ feet per second
    \end{itemize}
\end{enumerate}

\subsection*{Turning in the assignment}

\begin{itemize}
    \item Activity 1: Submit responses to the five parts at this Google Form: https://bit.ly/31Hg3dJ 
    \item Activity 2: Simply collaborate with your group in your private chat area on CampusWire; there's nothing separate to turn in. 
\end{itemize}

\subsection*{Grading}

Like all Post-Class activities in the course, this Post-Class activity is worth \textbf{2 engagement credits}: 

\begin{itemize}
    \item \textbf{2 engagement credits} are given for work that represents a good-faith effort to give correct responses on all activity items as well as substantive contributions to all online group activities, and turned in before the deadline. 
    \item \textbf{1 engagement credit} is given for work that has some positive effort involved and which is completed on time, but which has some deficiencies. These might include one or more blank responses to items, minimal contributions to online group discussions, or responses that do not represent a good-faith effort at demonstrating your understanding. 
    \item \textbf{0 engagement credits} are given for work that is mostly blank, that shows minimal effort, or is late; also 0 credits are given if you make no substantive contribution to online group discussions. 
\end{itemize}

Please note \textbf{mathematical correctness is not necessary to earn full credit.} You just need to engage actively with the assignment and with each other. 

\end{document}