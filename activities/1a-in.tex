\documentclass{beamer}

\usepackage[utf8]{inputenc}
%\usetheme{Warsaw}

%Information to be included in the title page:
\title{MTH 201: Calculus}
\subtitle{Module 1A: How do we measure velocity?}
\author{Prof. Talbert}
\institute{GVSU}
\date{\today}


\begin{document}

\frame{\titlepage}


\begin{frame}{Agenda for today}
    \begin{itemize}
        \item<1-> Review of Daily Prep assignment, and Q+A
        \item<2-> Activity: Going from average velocity to instantaneous velocity
        \item<3-> Minilecture: What this means from a graphical point of view 
        \item<4-> Further practice with the concept
        \item<5-> For next time: Followup activities and things to do 
    \end{itemize}
\end{frame}

\begin{frame}{Polling for today}
    \large{
    Go to \textbf{\url{www.mentimeter.com}} and enter code \texttt{xx yy zz} 
    }
    \end{frame}

\begin{frame}{A basic question}
    
    \begin{block}{Reminder}
        The \textbf{average velocity} of a moving object is an estimate of its velocity over an \textbf{interval} of time. The \textbf{instantaneous velocity} of a moving object is its velocity at a \textbf{single moment} in time. 
    \end{block}
    
    \begin{alertblock}{Fundamental Question}
        It's easy to find average velocity given two points. But how do you find instantaneous velocity, where you only have \emph{one} point? 
    \end{alertblock}
    
    Activity: On your device, go to the spreadsheet set up at: 
    
    \begin{center}
        https://bit.ly/201-1a
    \end{center}
    
    \end{frame}
    
\begin{frame}{Debrief with a graph}
        
\end{frame}
    
    \section[What to do next]{What to do next}
    \begin{frame}{Coming up next}
        \begin{itemize}
        \item Next up stuff here 
    \end{itemize}



\begin{frame}
    \frametitle{NEXT TIME...}

    \begin{itemize}
        \item To do list 1
        \item To do list 2
    \end{itemize}

\end{frame}

\end{document}