\documentclass{beamer}

\usepackage[utf8]{inputenc}
\usetheme{Warsaw}

%Information to be included in the title page:
\title{MTH 201: Calculus}
\subtitle{Followup Activities --- Module 1B}
\author{Prof. Talbert}
\institute{GVSU}
\date{\today}


\begin{document}

\frame{\titlepage}

\begin{frame}{Reminders:}

    \begin{itemize}
        \item Post-class activities are to be worked out in ClassKick.
        \item There is nothing to turn in --- your work is saved automatically on ClassKick. 
        \item They are graded check/x on the basis of completeness and effort. 
        \item You can work freely with others on these, but please for your own benefit, don't just copy work. 
        \item If you need help or want Prof. Talbert to check your work, use the ``raise hand" feature on ClassKick. 
    \end{itemize}
    
    \end{frame}

\begin{frame}[t]{Limits with algebra}

    Consider the limit: 
        $$\lim_{x \to 2} \frac{16-x^4}{x^2-4}$$
        
     First, use a table of values or a graph to estimate the limit. If you use a graph, include a link to the Desmos graph or embed an image of the graph here. 
\end{frame}
        
\begin{frame}[t]{Limits with algebra continued}
    Now use algebra to simplify the expression first and then find the limit. Hint: Start by factoring the top and the bottom. You can use WolframAlpha (www.wolframalpha.com) to help if you want. Note that your answer here should agree with the answer you got from the graph. 
        
    \end{frame}

\begin{frame}[t]{Average velocity}
        The position function for a falling ball is given by $s(t) = 16 - 16t^2$, where $s$ is measured in feet and $t$ in seconds. Write an expression that gives the \textbf{average} velocity of the ball from $t=1$ to $t = 1+h$ and simplify completely. 
\end{frame}
        
\begin{frame}[t]{Instantaneous velocity}
        Use the expression for average velocity to find the \textbf{instantaneous} velocity of the ball at the moment $t=1$. Spoiler: The answer is $-32$ feet per second. 
\end{frame}
        
\begin{frame}[t]{Reflecting}
    Overall, how comfortable do you feel with the concepts of this lesson? What questions, comments, or concerns do you have about Module 1B so far?     
        
        
    \end{frame}
    

\end{document}