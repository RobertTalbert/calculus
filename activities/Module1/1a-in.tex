\documentclass{beamer}

\usepackage[utf8]{inputenc}
%\usetheme{Warsaw}

%Information to be included in the title page:
\title{MTH 201: Calculus}
\subtitle{Module 1A: How do we measure velocity?}
\author{Prof. Talbert}
\institute{GVSU}
\date{\today}


\begin{document}

\frame{\titlepage}


\begin{frame}{Agenda for today}
    \begin{itemize}
        \item<1-> Review of Daily Prep assignment, and Q+A
        \item<2-> Activity: Going from average velocity to instantaneous velocity
        \item<3-> Minilecture: What this means from a graphical point of view 
        \item<4-> Further practice with the concept
        \item<5-> For next time: Followup activities and things to do 
    \end{itemize}
\end{frame}

\begin{frame}{Polling for today}
    \large{
    Go to \textbf{\url{www.mentimeter.com}} and enter code \texttt{xx yy zz} 
    }
    \end{frame}

\begin{frame}{A basic question}
    
    \begin{block}{Reminder}
        The \textbf{average velocity} of a moving object is an estimate of its velocity over an \textbf{interval} of time. The \textbf{instantaneous velocity} of a moving object is its velocity at a \textbf{single moment} in time. 
    \end{block}
    
    \begin{alertblock}{Fundamental Question}
        It's easy to find average velocity given two points. But how do you find instantaneous velocity, where you only have \emph{one} point? 
    \end{alertblock}
    
    Activity: On your device, go to the spreadsheet set up at: 
    
    \begin{center}
        \url{https://bit.ly/201-1a}
    \end{center}
    
    \end{frame}
    
\begin{frame}{Debrief with a graph}
     \begin{center}
        \url{https://www.desmos.com/calculator/obpytdzxsf}
    \end{center}
\end{frame}

\begin{frame}{Summing it up}
    \begin{itemize}
        \item The \textbf{average velocity} of an object on the interval $[a,b]$ is the \textbf{slope of the ``secant line''} that goes through $(a,s(a))$ and $(b, s(b))$. \pause
        \item \alert{To find the \textbf{instantaneous velocity} of the object at the \emph{single} time value $t = a$: Move the second point $b$ closer to $a$, measure the average velocity, and repeat -- look for a single value that is being approached.} \pause
        \item Alternate take: Think of the second point $b$ as $a + h$ where $h$ is a small distance, and let $h$ move toward zero. \pause
        \item \alert{It's also the slope of the ``tangent line'' that touches the graph of $s(t)$ at the single point $(a,s(a))$. If we zoomed in on the graph of $s$ at this point, the graph would flatten out at appear to be equal to this line.} \pause
    \end{itemize}
\end{frame}

\begin{frame}{Applying the concept}
    
    \begin{center}
        Back to Mentimeter for two polling questions
    \end{center}
    
\end{frame}

    




\begin{frame}
    \frametitle{NEXT TIME...}

    \begin{itemize}
        \item \textbf{Followup activities:} To be done on your schedule, posted to ClassKick (watch CampusWire and Blackboard for a link). Complete by due date for 1 engagement credit. 
        \item \textbf{Daily Prep for Part B:} Go ahead and start reading/watching video; see calendar for due date
        \item \textbf{Ask questions and interact:} Get on CampusWire and share thoughts, questions, and help. 
    \end{itemize}

\end{frame}

\end{document} 