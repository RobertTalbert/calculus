\documentclass{beamer}

\usepackage[utf8]{inputenc}


%Information to be included in the title page:
\title{MTH 201: Calculus}
\subtitle{Module 2A: The derivative of a function at a point}
\author{Prof. Talbert}
\institute{GVSU}
\date{\today}


\begin{document}

\frame{\titlepage}

\begin{frame}{Agenda for today}
    \begin{itemize}
        \item<1-> Review of Daily Prep assignment, and Q+A
        \item<2-> Short lecture: Computing a derivative with limits  
        \item<3-> Practice with computing derivatives using limits 
        \item<4-> Polling activity on the concept of the deriative 
        \item<5-> For next time: Followup activities and things to do 
    \end{itemize}
\end{frame}

\begin{frame}
    \frametitle{Review and Q+A}

    \begin{center}
        Go to \url{www.menti.com} and use code \texttt{?}
    \end{center}

\end{frame}

\begin{frame}
    \frametitle{Computing derivatives}

    \begin{alertblock}{The definition of the derivative}
        Let $f$ be a function and $x=a$ a value in the function's domain. We define the \textbf{derivative of $f$ with respect to $x$ at evaluated at $x=a$}, denoted $f'(a)$, by the formula
        $$f'(a) = \lim_{h \to 0} \frac{f(a+h) - f(a)}{h}$$
        provided the limit exists. 
    \end{alertblock}

    What the formula means: 
    \begin{itemize}
        \item $\frac{f(a+h) - f(a)}{h}$ is the \emph{average rate of change} in $f$ on an interval starting at $x = a$ and ending at $x = a + h$ ($h$ is the length of the interval)
        \item $\lim_{h \to 0} \frac{f(a+h) - f(a)}{h}$ is what happens to those average rates as the length of the interval shrinks to 0. 
    \end{itemize}

\end{frame}


\begin{frame}
    \frametitle{Example}

    \begin{exampleblock}{Example}
        Let $f(x) = x^2 - 2x + 1$. Find the value of $f'(2)$ using the definition. \\

        (See whiteboard for solution)
    \end{exampleblock}  \pause

    \begin{align*}
        f'(2) &= \lim_{h \to 0} \frac{f(2+h) - f(2)}{h} \\
        &= \lim_{h \to 0} \frac{((2+h)^2 - 2(2+h) + 1) - (2^2 - 2(2) + 1)}{h} \\
        &= \lim_{h \to 0} \frac{(4+4h+h^2 - 4 - 2h + 1)-(1)}{h} \\
        &= \lim_{h \to 0} \frac{2h+h^2}{h} \\
        &= \lim_{h \to 0} (2 + h) \\
        &= 2. 
    \end{align*}

Desmos: Does the answer make sense? 

\end{frame}

\begin{frame}
    \frametitle{Bonus: $f'(1)$}

    Replace all the 2's with 1's, basically

    \begin{align*}
        f'(\textcolor{red}{1}) &= \lim_{h \to 0} \frac{f(\textcolor{red}{1}+h) - f(\textcolor{red}{1})}{h} \\
        &= \lim_{h \to 0} \frac{((\textcolor{red}{1}+h)^2 - 2(\textcolor{red}{1}+h) + 1) - (\textcolor{red}{1}^2 - 2(\textcolor{red}{1}) + 1)}{h} \\
        &= \lim_{h \to 0} \frac{(1+2h + h^2 - 2 - 2h + 1)-(0)}{h} \\
        &= \lim_{h \to 0} \frac{h^2}{h} \\
        &= \lim_{h \to 0} h \\
        &= 0.
    \end{align*}
    
    Desmos: Does the answer make sense? 

\end{frame}

\begin{frame}
    \frametitle{In groups}

    Let $f(x) = 3 - 2x$. \\

    \begin{enumerate}
        \item Set up the limit that would compute $f'(5)$. 
        \item Before calculating the limit, go to Desmos and graph $f$. What \emph{should} the value of $f'(5)$ be, and why? 
        \item Now go through the limit computation step by step with your partner(s). Does your result verify your guess? 
    \end{enumerate}

\end{frame}

\begin{frame}
    \frametitle{Bonus practice}

    This will appear in your follow-up activity. If you have time, you can get started here. 

    \begin{block}{Velocity}
        A water balloon is tossed vertically in the air from a window. The balloon's height in feet at time $t$ in seconds after being launched is given by $s(t) = -16t^2 + 16t + 32$. 

        \begin{itemize}
            \item Set up the limit that will compute the instantaneous velocity of the balloon at time $t=1$. 
            \item Graph $s(t)$ on Desmos and estimate what the value of the instantaneous velocity should be. 
            \item Now compute the limit you set up to find the \emph{exact} value of the velocity. 
        \end{itemize}
    \end{block}

\end{frame}

\begin{frame}
    \frametitle{Next}

\textbf{All due dates are on the Course Calendar}


    \begin{itemize}
        \item Complete Followup Activities
    \end{itemize}

\end{frame}
 
\end{document}