\documentclass{beamer}

\usepackage[utf8]{inputenc}


%Information to be included in the title page:
\title{MTH 201: Calculus}
\subtitle{Module 2B: The derivative of a function at a point}
\author{Prof. Talbert}
\institute{GVSU}
\date{\today}


\begin{document}

\frame{\titlepage}

\begin{frame}{Agenda for today}
    \begin{itemize}
        \item<1-> Review of Daily Prep assignment, and Q+A
        \item<2-> Short lecture: The derivative as a function 
        \item<3-> Practice with computing derivatives using limits 
        \item<4-> Polling activity: Identifying derivative graphs 
        \item<5-> For next time: Followup activities and things to do 
    \end{itemize}
\end{frame}

\begin{frame}
    \frametitle{Review and Q+A}

    \begin{center}
        Go to \url{www.menti.com} and use code \texttt{?}
    \end{center}

\end{frame}

\begin{frame}
    \frametitle{The derivative as a function}

    Last time: A formula for finding the derivative of a function at a single point

    \begin{alertblock}{The definition of the derivative}
        Let $f$ be a function and $x=a$ a value in the function's domain. We define the \textbf{derivative of $f$ with respect to $x$ at evaluated at $x=a$}, denoted $f'(a)$, by the formula
        $$f'(a) = \lim_{h \to 0} \frac{f(a+h) - f(a)}{h}$$
        provided the limit exists. 
    \end{alertblock}

    $f'(a)$ tells us: 
    \begin{itemize}
        \item The slope of the tangent line to the graph of $f$ at $x=a$
        \item The instantaneous rate of change in $f$ at $x=a$
        \item If $f$ is a position: $f'(a)$ is the instantaneous velocity at time $x=a$
    \end{itemize}

\end{frame}

\begin{frame}
    \frametitle{The derivative is a function on its own}

    \begin{block}{Key insight}
        The derivative itself changes as $a$ changes. So \alert{not only is $f$ a function of $x$, so is $f'$.}
    \end{block}
    
    \begin{exampleblock}{Example}
        Let $f(x) = x-x^2$. Then: 

        \begin{tabular}{c||c|c|c|c|c|c|c}
            $a$ & $-3$ & $-2$ & $-1$ & $0$ & $1$ & $2$ & $3$ \\ \hline
            $f(a)$ & $-12$ & $-6$ & $-2$ & $0$ & $0$ & $-2$ & $-6$ \\ \hline
            $f'(a)$ & $7$ & $5$ & $3$ & $1$ & $-1$ & $-3$ & $-5$ 
        \end{tabular}

    \end{exampleblock}

    $$f'(3) = \lim_{h \to 0} \frac{f(3+h)-f(3)}{h} = \lim_{h \to 0} \frac{((3+h) - (3+h)^2)-(-6-(-6)^2))}{h}$$

    \alert{Do we have to recalculate $f'(a)$ using a limit \emph{every time}?} Also do you see the pattern in the $f'(a)$ values? 

\end{frame}


\begin{frame}
    \frametitle{Computing derivative FORMULAS}

    It looks like, if $f(x) = x-x^2$ then $\textcolor{red}{f'(x) = 1-2x}$. Does this always work? \pause 

    \begin{align*}
        f'(x) &= \lim_{h \to 0} \frac{f(\textcolor{red}{x}+h) - f(\textcolor{red}{x})}{h} \\ 
        &= \lim_{h \to 0} \frac{((x+h) - (x+h)^2)-(x - x^2)}{h} \\
        &= \lim_{h \to 0} \frac{(x+h - (x^2 + 2xh + h^2)) - (x - x^2)}{h} \\
        &= \lim_{h \to 0} \frac{x+h - x^2 - 2xh - h^2 -x + x^2}{h} \\
        &= \lim_{h \to 0} \frac{h-2xh - h^2}{h} \\
        &= \lim_{h \to 0} (1 - 2x - h) \\
        &= 1-2x
    \end{align*}

\end{frame}

\begin{frame}

    \begin{alertblock}{The definition of the derivative (as a formula)}
        Let $f$ be a function and $x=a$ a value in the function's domain. We define the \textbf{derivative of $f$ with respect to $x$}, denoted $f'(x)$, by the formula
        $$f'(x) = \lim_{h \to 0} \frac{f(x+h) - f(x)}{h}$$
        provided the limit exists. 
    \end{alertblock}

The result is a \alert{formula} that accepts $x$ as an input and gives the \alert{rate of change in $f$} at this input, as the output. \\

\textbf{Demo}: https://www.desmos.com/calculator/rwjzrvo9an 

\end{frame}


\begin{frame}

    \begin{exampleblock}{Practice}
        Let $f(x) = x^2 - 2x + 1$. Find a formula for $f'(x)$ using the definition. 
    \end{exampleblock}  \pause

    \begin{align*}
        f'(x) &= \lim_{h \to 0} \frac{f(x+h) - f(x)}{h} \\
        &= \lim_{h \to 0} \frac{((x+h)^2 - 2(x+h) + 1) - (x^2 - 2x + 1)}{h} \\
        &= \lim_{h \to 0} \frac{(x^2 + 2xh + h^2 - 2(x+h) + 1)- (x^2 - 2x + 1)}{h} \\
        &= \lim_{h \to 0} \frac{x^2 + 2xh + h^2 - 2x - 2h + 1 - x^2 + 2x - 1}{h} \\
        &= \lim_{h \to 0} \frac{2xh + h^2 - 2h}{h} \\
        &= \lim_{h \to 0} (2x + h  - 2) \\
        &= 2x - 2 
    \end{align*}

\textbf{Desmos:} Does the answer make sense? 

\end{frame}

\begin{frame}
    \frametitle{Information about this idea}

    \begin{itemize}
        \item We will eventually shorten this process considerably with \alert{shortcut methods}. 
        \item But you still need to know the definition, because \alert{not all functions are formulas}. 
        \item Being able to do these computations is a Core Learning Target (D.1). 
    \end{itemize}

\end{frame}

\begin{frame}
    \frametitle{Derivatives of formulas as graphs}

    \textbf{Back to Menti:} Can you identify the graph of the derivative given the graph of the function and vice versa? 

\end{frame}


\begin{frame}
    \frametitle{Next}

\textbf{All due dates are on the Course Calendar}


    \begin{itemize}
        \item Complete Followup Activities
    \end{itemize}

\end{frame}
 
\end{document}