\documentclass{beamer}

\usepackage[utf8]{inputenc}
%\usetheme{Warsaw}

%Information to be included in the title page:
\title{MTH 201: Calculus}
\subtitle{Module 1A: Followup activities}
\author{Prof. Talbert}
\institute{GVSU}
\date{\today}


\begin{document}

\frame{\titlepage}

\begin{frame}{Reminders:}

\begin{itemize}
    \item Post-class activities are to be worked out in ClassKick.
    \item \textbf{These are not optional}! They contain key concepts that you will be tested on later. 
    \item There is nothing to turn in --- your work is saved automatically on ClassKick. 
    \item \textbf{They are graded check/x on the basis of completeness and effort}, like Daily Prep activities. A check counts as 1 engagement credit. 
    \item You can work freely with others on these, but please for your own benefit, don't just copy work. 
    \item If you need help or want Prof. Talbert to check your work, use the "raise hand" feature on ClassKick. 
\end{itemize}

\end{frame}


\begin{frame}[t]{1: Average velocity, alternate take}
The position function for a falling ball is $s(t)=64-16(t-1)^2$, with $t$ in seconds and $s$ in feet. Calculate an expression for the average velocity of the ball on the interval $[2,2+h]$. Do algebra to completely simplify the resulting expression. Show your work below (or in a picture you upload and embed here). 
    
\end{frame}

\begin{frame}[t]{2: Using the average velocity expression}
Take the simplified expression you came up with and use it to find the average velocity of the ball on the interval $[2, 2.5]$. Show your work. \textit{Spoiler}: The answer is $-40$ feet per second. If you come up with something else, debug your work on the previous slide as well as on this one. 

\end{frame}

\begin{frame}[t]{3: Getting to instantaneous velocity}
    Use the expression from part 1 of this activity to find the instantaneous velocity of the ball at $t=2$. Explain your reasoning. \textit{Hint}: Think about what you should do with the variable $h$. 
\end{frame}

\begin{frame}[t]{Reflecting}
Overall, how comfortable do you feel with the concepts of this lesson? What questions, comments, or concerns do you have about Module 1A so far?     
    
    
\end{frame}

\end{document} 