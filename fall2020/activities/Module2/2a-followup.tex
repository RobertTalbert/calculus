\documentclass{beamer}

\usepackage[utf8]{inputenc}
\usetheme{Warsaw}

%Information to be included in the title page:
\title{MTH 201: Calculus}
\subtitle{Followup Activities --- Module 2A}
\author{Prof. Talbert}
\institute{GVSU}



\begin{document}

%\frame{\titlepage}

\begin{frame}{Reminders:}

    \begin{itemize}
        \item Post-class activities are to be worked out in ClassKick.
        \item There is nothing to turn in --- your work is saved automatically on ClassKick. 
        \item They are graded check/x on the basis of completeness and effort. 
        \item You can work freely with others on these, but please for your own benefit, don't just copy work. 
        \item If you need help or want Prof. Talbert to check your work, use the "raise hand" feature on ClassKick. 
    \end{itemize}
    
    \end{frame}

    \begin{frame}
        \frametitle{Connecting derivatives and velocity}
    
        This activity goes through the following problem step by step: 
    
        \begin{block}{Velocity}
            A water balloon is tossed vertically in the air from a window. The balloon's height in feet at time $t$ in seconds after being launched is given by $s(t) = -16t^2 + 16t + 32$. 
    
           What is the instantaneous velocity of the balloon at $t=1$ second? 

        \end{block}
    
    \end{frame}

    \begin{frame}[t]
        \frametitle{}
    
        Set up --- but do not yet evaluate --- the limit that will compute the instantaneous velocity of the balloon at time $t=1$. 
    
    \end{frame}

    \begin{frame}[t]
        \frametitle{}
    
        Graph $s(t)$ on Desmos upload an image of your graph. On the graph, draw the point $(1, s(1))$ and the tangent line to the graph of $s$ at $t=1$. Then estimate what you think the value of the instantaneous velocity of the balloon at $t=1$, and explain your reasoning. 
    
    \end{frame}

    \begin{frame}[t]
        \frametitle{}
        Now compute the limit you set up to find the \emph{exact} value of the velocity of the ball at $t=1$. The answer should agree with your estimate from the last slide; if not, debug your work. 
        
    
    \end{frame}

    \begin{frame}[t]{Reflecting}
        Overall, how comfortable do you feel with the concepts of this lesson? What questions, comments, or concerns do you have about Module 1B so far?     
            
            
        \end{frame}


\end{document}