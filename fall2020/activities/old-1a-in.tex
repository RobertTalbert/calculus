\documentclass[10pt]{beamer}

\usetheme[progressbar=frametitle]{metropolis}
% \usepackage{appendixnumberbeamer}

% \usepackage{booktabs}
% \usepackage[scale=2]{ccicons}

% \usepackage{pgfplots}
% \usepgfplotslibrary{dateplot}

% \usepackage{xspace}
% \newcommand{\themename}{\textbf{\textsc{metropolis}}\xspace}

\title{MTH 201: Calculus -- Module 1A}
\subtitle{How do we measure velocity? (\emph{Active Calculus} 1.1)}
% \date{\today}
\date{}
\author{Robert Talbert}
\institute{Grand Valley State University}
% \titlegraphic{\hfill\includegraphics[height=1.5cm]{logo.pdf}}

\begin{document}

\maketitle

\begin{frame}{Agenda}
  \setbeamertemplate{section in toc}[sections numbered]
  \tableofcontents%[hideallsubsections]
\end{frame}





 \section[Review of Daily Prep]{Daily Prep Review}

 \begin{frame}{Polling for today}
 \large{
 Go to \textbf{\url{www.mentimeter.com}} and enter code \texttt{xx yy zz} 
 \vskip
 Or  go to \url{https://www.menti.com/4d48wn64k6}}
 \end{frame}

\section[Q+A]{Q+A from Daily Prep}

\section[From average to instantaneous velocity]{From average to instantaneous velocity}

\begin{frame}{A basic question}
    \metroset{block=fill}

\begin{block}{Reminder}
    The \textbf{average velocity} of a moving object is an estimate of its velocity over an \textbf{interval} of time. The \textbf{instantaneous velocity} of a moving object is its velocity at a \textbf{single moment} in time. 
\end{block}

\begin{alertblock}{Fundamental Question}
    It's easy to find average velocity given two points. But how do you find instantaneous velocity, where you only have \emph{one} point? 
\end{alertblock}

Activity: On your device, go to the spreadsheet set up at: 

\begin{center}
    https://bit.ly/201-1a
\end{center}

\end{frame}

\begin{frame}{Debrief with a graph}
    
\end{frame}

\section[What to do next]{What to do next}
\begin{frame}{Coming up next}
    \begin{itemize}
    \item Next up stuff here 
\end{itemize}

\textbf{Continue to check email and announcements every day --- ask questions and give help on CampusWire too.}


\end{frame}


\end{document}
