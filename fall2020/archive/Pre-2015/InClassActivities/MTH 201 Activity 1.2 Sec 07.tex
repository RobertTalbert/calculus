\documentclass[11pt]{article}

% \pagestyle{empty}                       %no page numbers
% \thispagestyle{empty}                   %removes first page number
\setlength{\parindent}{0in}               %no paragraph indents

\usepackage{fullpage}
\usepackage[tmargin = 0.5in, bmargin = 1in, hmargin = 1in]{geometry}     %1-inch margins
\geometry{letterpaper}                  
\usepackage{graphicx}
\usepackage{amssymb}

% Default packages
\usepackage{latexsym}
\usepackage{amsfonts}
\usepackage{amsmath}
\usepackage{amsthm}
\usepackage{hyperref}
\usepackage{fancyhdr}
\usepackage{enumitem}
\usepackage{pifont}

\newcommand{\cuthere}{%
\noindent
\raisebox{-2.8pt}[0pt][0.75\baselineskip]{\small\ding{34}}
\unskip{\tiny\dotfill}
}

\def\ra{\rightarrow}

\def\pageturn{\vfill 
\begin{flushright}
	\begin{small}
		Continued $\ra$
	\end{small}
\end{flushright} \newpage}


\begin{document}
	
	\thispagestyle{empty}
	\renewcommand{\headrulewidth}{0.0pt}
	\thispagestyle{fancy}
	\lhead{Prof. Talbert}
	\chead{MTH 201: Calculus 1}
	\rhead{September 5, 2013}
	\lfoot{}
	\cfoot{}
	\rfoot{}	
	
	\vspace*{0in}

		\begin{center}
			\begin{large}
			\textbf{Class Activities: The notion of limit} \\
			\end{large}
		\end{center}
	
Get into groups of 2--4 and work through all of the following activities. These are not to be turned in, and they will not be graded. Instead, record your group's work on your copy and keep it for notes. Include enough detail in your work that you can use this worksheet for study purposes later. I will be coming to each group one by one as you work to observe what you're doing, answer questions, and catch any misconceptions that are happening. We will stop with about 10 minutes remaining to debrief the main ideas. 


\section{Taking limits numerically and algebraically}

\begin{enumerate}
	\item Consider the function $f(x) = \dfrac{x^2 - 1}{x - 1}$. Does the value of $f(1)$ exist? Why/why not?
	
	\vspace{0.3in}
	
	\item Using the same function $f(x)$ as above, use a calculator or spreadsheet to fill in the blanks of the following table. Some of these have already been done for you; your calculations should not be drastically different from the ones already here. For example a calculation equal to 22.5 is very likely incorrect. 
	\begin{center}
		\begin{tabular}{c|c||c|c}
		$x$ & $f(x)$ & $x$ & $f(x)$ \\ \hline
		$0.5$ & 1.5 & $1.5$ & \hspace{1in} \\
		$0.9$ & \hspace{1in} & $1.1$ & 2.1 \\
		$0.99$ & 1.99 & $1.01$ & \hspace{1in} \\
		$0.999$ & \hspace{1in} & $1.001$ & 2.001 \\
		\end{tabular}
	\end{center}
Based on the table, estimate the value of $\displaystyle{\lim_{x \to 1} \frac{x^2 - 1}{x-1}}$. (\textbf{NOTE}: The limit as $x \to 1$ exists even though the function at $x=1$ does not exist.)

	\item Use algebra to simplify the fraction $\dfrac{x^2 - 1}{x - 1}$. Use the result to find the \emph{exact} value of $\displaystyle{\lim_{x \to 1} \frac{x^2 - 1}{x-1}}$.  
	
	
	
\end{enumerate}

%%%%%

\vfill

\section{Average velocity and geometry}

Consider the position function of a flying ball given by $s(t) = 64 - 16(t-1)^2$, where $t$ is measured in seconds and $s$ is measured in feet. This is the same position function you worked with in the Preview Activity. Below, you're given a graph of this position function with two points shown on the graph. Point $A$ is at the point $(0.4, s(0.4))$ and point $B$ is at $(0.8, s(0.8))$. 

\pageturn

\begin{center}
	\includegraphics{act11-plot}
\end{center}

\begin{enumerate}
	\item Compute the average velocity of the ball on the interval $[0.4, 0.8]$. Show enough work that you can review this later for study purposes. Include correct units of measurement on your answers. Your average velocity should be an \emph{exact} value, \emph{not estimated off the graph}.
	
	\vspace{1in}
	
	\item On the graph, draw the line connecting points $A$ and $B$. Then calculate that line's slope. As with average velocity, your slope should be an \emph{exact} value, \emph{not estimated off the graph}.
	
	\item What is the relationship between the results of the previous two questions? 
	
	\vspace{0.3in}
	
	\item Fill in the blanks: 
	\begin{quote}
		The average velocity of the ball on the interval $[a,b]$ is the s\underline{\hspace{1in}} of the l\underline{\hspace{1in}} connecting the points 
		(\underline{\hspace{0.25in}}, \underline{\hspace{0.25in}}) and (\underline{\hspace{0.25in}}, \underline{\hspace{0.25in}}). 
 	\end{quote} 
\end{enumerate}

%%%%%

\section{Instantaneous velocity and limits}

Consider a moving object whose position function is given by $s(t) = t^2$ where $s$ is measured in meters and $t$ is measured in minutes. 

\begin{enumerate}
	\item Find an expression for the average velocity of the object on the interval $[4, 4+h]$. Simplify the expression algebraically as much as possible. (There should be no fractions present once the expression is fully simplified, just a simple expression that involves the variable $h$.)

\pageturn
	
	\item Use the result of question (1) to determine the average velocity of the object on the interval $[4, 4.02]$. Include units on your answer. (Hint: What's the value of $h$ in this case?) 
	
\vspace{1in}
	
	\item Determine the exact value of the instantaneous velocity of the object when $t=4$ by taking a limit of the average velocity formula you found in question (1). Include units on your answer. Important question: When you take a limit, what's your variable and what should it be approaching? 
	
\vspace{1.5in}	
	
	\item How would you go about finding the object's instantaneous velocity at $t = 10$ seconds? What would do differently in the above process? 

	
\end{enumerate}

\section{Bonus Round: More limits and velocities}
If we run out of time for this today, please work this out at home. \\


In the second part of this activity, a ball was moving with position given by $s(t) = 64 - 16(t-1)^2$, where $t$ is measured in seconds and $s$ is measured in feet. 
\begin{enumerate}
	\item Set up and fully simplify an algebraic expression for the average velocity of the ball on the interval $[2, 2+h]$. The result should have no fractions in the expression at all.
	\item Take a limit of the resulting expression to find the exact value of the instantaneous velocity of the ball at the time $t = 2$. (What limit should you take?) \emph{The answer should come out to $-32$ feet per second. }
\end{enumerate}

If you get done with that, suppose the ball is moving with position function $s(t) = 100 + 10t - 16t^2$ and find its instantaneous velocity at $t = 1$, using a limit of an average velocity. (If you have had calculus before and know shortcuts, put those away for now and use only algebra and limits.) The answer here should come out to $-22$. 


% \section{Average velocity, instantaneous velocity, and slopes}
% 
% A moving object has position $s$ at time $t$ given by the graph below: 
% \begin{center}
% 	\includegraphics[width=0.5\textwidth]{act1-2}
% \end{center}
% 
% Assume $s$ is measured in feet and $t$ is in seconds. 
% 
% \pageturn
% 
% \begin{enumerate}
% 	\item Use the graph to estimate the average velocity of the object on the intervals $[2,3]$, $[2,2.5]$, and $[2, 2.25]$. For each interval, draw a line on the graph whose slope is the average velocity on that interval. For example, the average velocity on the interval $[1,2]$ is $\frac{s(2)-s(1)}{2-1}$, which is also the slope of the line connecting $(1, s(1))$ and $(2, s(2))$ 
% 		
% 	\vspace{2in}
% 	
% 	\item Use the graph to estimate the instantaneous velocity of the object when $t=2$. 
% 	
% 	\vspace{1.5in}
% 	
% 	\item Is the instantaneous velocity of the object at $t=2$ greater than, less than, or equal to the average velocity on the interval $[2,3]$? Explain. 
% \end{enumerate}




\vfill

\cuthere

\noindent
\textbf{What was the least clear point from today's class?}

\vspace{1in}

\end{document}