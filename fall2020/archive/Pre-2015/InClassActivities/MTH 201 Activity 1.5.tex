\documentclass[11pt]{article}

\pagestyle{empty}                       %no page numbers
\thispagestyle{empty}                   %removes first page number
\setlength{\parindent}{0in}               %no paragraph indents

\usepackage{fullpage}
\usepackage[tmargin = 0.5in, bmargin = 1in, hmargin = 1in]{geometry}     %1-inch margins
\geometry{letterpaper}                  
\usepackage{graphicx}
\usepackage{amssymb}

% Default packages
\usepackage{latexsym}
\usepackage{amsfonts}
\usepackage{amsmath}
\usepackage{amsthm}
\usepackage{palatino}
\usepackage{hyperref}
\usepackage{multicol}
\usepackage{fancyhdr}
\usepackage{enumitem}

\def\ra{\rightarrow}

\def\pageturn{\vfill 
\begin{flushright}
	\begin{small}
		Continued $\ra$
	\end{small}
\end{flushright} \newpage}


\begin{document}
	
	\thispagestyle{empty}
	\renewcommand{\headrulewidth}{0.0pt}
	\thispagestyle{fancy}
	\lhead{Prof. Talbert}
	\chead{MTH 201: Calculus 1}
	\rhead{May 7, 2013}
	\lfoot{}
	\cfoot{}
	\rfoot{}	
	
	\vspace*{0in}

		\begin{center}
			\begin{large}
			\textbf{Class Activity: Limits and Instantaneous Velocity} \\
			\end{large}
		\end{center}
		
This is Activity 1.5 in your book, with a little more structure placed around it. Please work in groups of 1--3. You do NOT need to turn this in, so just use this handout as a template for notes. Do, however, ask questions if you have any. \\

In the following we are working with the function 
\[ s(t) = t^2 \]
which models the position (in meters) of a moving object at time $t$ minutes. 

\begin{enumerate}

\item Determine a simplified expression for the average velocity of the object on the interval $[3, 3+h]$ by first recalling that 
\[ AV_{[a,a + h]} = \frac{s(a + h) - s(a)}{h} \]
	\begin{enumerate}
		\item What is the value of $a$ here? Is the value of $h$ a fixed number, or a variable? 
		
		\vspace{0.5in}
		
		\item Calculate and fully simplify the expression $s(3+h)$. 
		
		\vspace{1in}
		
		\item Calculate $s(3)$ 
		
		\vspace{0.5in}
		
		\item Calculate and simplify $s(3+h) - s(3)$. 
		
		\pageturn
		
		\item The average velocity on $[3, 3+h]$ is now the fraction given in the formula above. Calculate and fully simplify this fraction. There should be no fractions in your result --- watch your algebra. 
	\end{enumerate}
	
	
\vspace{2in}	
	
\item Use your result from the previous question to find the average velocity of the object on the interval $[3, 3.2]$, and include units on your answer. 

\vspace{1in}


\item Determine the instantaneous velocity at $t=3$ by filling the values on these tables and then using the results: 

\begin{center}
	\begin{tabular}{c|c}
	Time interval & AV on this time interval \\ \hline
	$[2.8, 3]$ &  \\
	$[2.9, 3]$ &  \\
	$[2.99, 3]$ &  \\
	$[2.999, 3]$ &  \\
	\end{tabular}
	\qquad 
	\begin{tabular}{c|c}
	Time interval & AV on this time interval \\ \hline
	$[3, 3.2]$ &  \\
	$[3, 3.1]$ &  \\
	$[3, 3.01]$ &  \\
	$[3, 3.001]$ &  \\
	\end{tabular}	
\end{center}



\end{enumerate}

\end{document}