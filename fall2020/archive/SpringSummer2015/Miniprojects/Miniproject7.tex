\documentclass[11pt,letterpaper]{article}

\usepackage{fancyhdr}
\usepackage[latin1]{inputenc}
\usepackage{amsmath}
\usepackage{amsfonts}
\usepackage{amssymb}
\usepackage{graphicx}
\usepackage[hmargin=2cm,vmargin=2.5cm]{geometry}
\usepackage[normalem]{ulem}
\usepackage{enumerate}
\usepackage{hyperref}



\pagestyle{fancy}
\setlength\parindent{0in}
\setlength\parskip{0.1in}
\setlength\headheight{15pt}


\begin{document}

\begin{flushright}
	\begin{Large}
		Miniproject 7: Related Rates
	\end{Large}
\end{flushright}

\noindent
\textbf{Overview:} This miniproject focuses on another important class of applied calculus problems, namely \emph{related rates} problems. These problems augment and extend the kinds of problems you have worked with in WeBWorK and class discussions. 

\medskip

\noindent
\textbf{Prerequisites:} Section 3.5 of \emph{Active Calculus.}

\medskip

\noindent
This miniproject is \textbf{not} one of the CORE miniprojects. This miniproject is an elective for those targeting a course grade of A or B. 	

\hrulefill

For this miniproject, select EXACTLY TWO of the following and give complete and correct solutions that abide by the specifications for student work. 

\begin{description}
	\item[Problem 1.] A sailboat is sitting at rest near its dock. A rope attached to the bow of the boat is drawn in over a pulley that stands on a post on the end of the dock that is 5 feet higher than the bow. If the rope is being pulled in at a rate of 2 feet per second, how fast is the boat approaching the dock when the length of rope from bow to pulley is 13 feet?
	\item[Problem 2.] A baseball diamond is a square with sides 90 feet long. Suppose a baseball player is advancing from second to third base at the rate of 24 feet per second, and an umpire is standing on home plate. Let $\theta$ be the angle between the third baseline and the line of sight from the umpire to the runner. How fast is $\theta$ changing when the runner is 30 feet from third base?
	\item[Problem 3.] Sand is being dumped off a conveyor belt onto a pile in such a way that the pile forms in the shape of a cone whose radius is always equal to its height. Assuming that the sand is being dumped at a rate of 10 cubic feet per minute, how fast is the height of the pile changing when there are 1000 cubic feet on the pile?
\end{description}


\hrulefill

\noindent
\textbf{Submission instructions:} Please prepare a writeup that includes all your work for two of the above problems. Remember to begin each solution with a clear statement of the problem. 

The resulting writeup can be done in whatever fashion you wish but it must be saved as a PDF file and submitted using Blackboard. (You may use any program you want to write the writeup but the submission \emph{must} be a PDF, or your work will be marked at Novice level and returned without comment.) 

\noindent
\textbf{File name:} Please give your PDF file for this miniproject the title: 
\begin{verbatim}
	LastName Miniproject7.pdf
\end{verbatim}
where \texttt{LastName} is your last name. 

\end{document}