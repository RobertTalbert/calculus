\documentclass[11pt,letterpaper]{article}

\usepackage{fancyhdr}
\usepackage[latin1]{inputenc}
\usepackage{amsmath}
\usepackage{amsfonts}
\usepackage{amssymb}
\usepackage{graphicx}
\usepackage[hmargin=2cm,vmargin=2.5cm]{geometry}
\usepackage[normalem]{ulem}
\usepackage{enumerate}
\usepackage{hyperref}



\pagestyle{fancy}
\setlength\parindent{0in}
\setlength\parskip{0.1in}
\setlength\headheight{15pt}


\begin{document}

\begin{flushright}
	\begin{Large}
		Miniproject 2: Graphical and symbolic modeling and calculus
	\end{Large}
\end{flushright}

\noindent
\textbf{Overview:} In this miniproject you will use technological tools to work with sets of data and build models of real-world quantitative phenomena, then apply the principles of the derivative to them to extract information about how the quantitative relationship changes -- just like you did in the first miniproject, but this time using graphical and analytical analysis instead of numerical. 

\medskip

\noindent
\textbf{Prerequisites:} Sections 1.1--1.5 in \emph{Active Calculus}, specifically graphical interpretation of the derivative and using the limit-based definition of the derivative to calculate instantaneous rate of change. Also the use of spreadsheets and Geogebra. Finally, \textbf{completion of Miniproject 1 is recommended before doing this miniproject} because some of the data used in Miniproject 1 are re-used here. 

\medskip

\noindent
\textbf{This miniproject is one of the four designated CORE miniprojects.} Successful completion of this miniproject with a \textbf{Mastery} rating is required to earn a C or above in the course. 
	

\hrulefill

In Miniproject 1 in the second task, you chose a set of two-variable data and analyzed the derivative of the data numerically. In Miniproject 2, we're going to keep that same data set and look at the derivative from both graphical and algebraic viewpoints. If you want to change your data set, you may do so as long as it's not population/time data as you saw in Miniproject 1 Task 1. Please let me know somehow on the writeup for Miniproject 2 if you choose to switch your data set. 

\medskip

\noindent
\textbf{Before working on this project:} Go and watch the following videos: 
\begin{itemize}
	\item How to create a scatterplot with trendline in Google Spreadsheets: \url{http://www.youtube.com/watch?v=tDy3g6Bw6gY} (Running time 2:13) 
	\item Plotting and fitting data in Geogebra: \url{https://www.youtube.com/watch?v=aeV1gjd2o-U} (Running time 4:56) 
\end{itemize}

\bigskip

\begin{enumerate}
	\item Start up a new Google Spreadsheet and put your original data set from Miniproject 1, Task 2 into this new spreadsheet. Do \emph{not} include the derivative data -- just the original data. You should be able to just copy and paste the data from the old spreadsheet into the new one. 
		\begin{enumerate}
			\item Following the instructions in the first video above, create a scatterplot for your data and display a trendline. \textbf{Experiment with different types of trendlines until you find one that is not linear that fits your data especially well.}
			\item Using the trendline, a straightedge and the gridlines on your chart, estimate the derivative of your data at the rightmost data point. 
		\end{enumerate}
	As you did in Miniproject 1, create a link to your spreadsheet that will allow me to comment directly on your spreadsheet. 

	\item Now take your data and put them into the Geogebra spreadsheet view (you should be able to copy and paste from your Google Spreadsheet). Using the instructions in the second video above, fit your data with a \emph{second-degree polynomial} and copy down the formula that you are given. By hand and using only the limit-based definition of the derivative, find the derivative of your data at the rightmost data point. \textbf{Show all work} -- working by hand is OK but you must adhere to the specifications for formatting in the ``Specifications for student work'' document. Note that you can check your result in at least three ways: By comparing the result with your work in the first task on this miniproject; by comparing your work with the derivative value you found numerically in Miniproject 1; and by having Geogebra calculate the derivative value. 

\end{enumerate}

\hrulefill

\noindent
\textbf{Submission instructions:} Please prepare a writeup that includes the following: 
	\begin{itemize}
		\item The link to your spreadsheet in task 1. Make sure you set the sharing settings so that I can comment, and not just view. 
		\item The result of the second part of task 1 (finding the derivative graphically) and a brief explanation of how you got your answer. 
		\item Complete work on the derivative process in task 2. 
	\end{itemize}
The resulting writeup can be done in whatever fashion you wish but it must be saved as a PDF file and submitted using Blackboard. (You may use any program you want to write the writeup but the submission \emph{must} be a PDF, or your work will be marked at Novice level and returned without comment.) 

\noindent
\textbf{File name:} Please give your PDF file for this miniproject the title: 
\begin{verbatim}
	LastName Miniproject2.pdf
\end{verbatim}
where \texttt{LastName} is your last name. 

\end{document}